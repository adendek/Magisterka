\chapter*{Wstęp}
\addcontentsline{toc}{chapter}{Wstęp}

Od zarania dziejów człowiek starał się zrozumieć pochodzenie świata, chciał odpowiedzieć na fundamentalne pytania ``Co nas otacza? '', ``Z czego to jest zbudowane?''. Obecnie Fizyka Wysokich Energii (ang. High Energy Physics HEP) kontynuuje badania w celu znalezienia podstawowych bloków budujących materię oraz ich wzajemnych oddziaływań.

Według obecnego stanu wiedzy przyjmuje się, że cała materia zbudowana jest z sześciu kwarków (u,d,c,s, t,b) oraz sześciu leptonów ($e, \mu, \tau, \nu_e,\nu_{\mu},\nu_{\tau}$). Każdemu kwarkowi i leptonowi odpowiada antycząstka. Antycząstki zostały przewidziane przez Paula Diraca podczas próby połączenia Mechaniki Kwantowej ze Szczególną Teorią Względności, której owocem jest słynne równanie Diraca. Cząstki oddziałują ze sobą poprzez wymianę bozonów pośredniczących. Znane są cztery fundamentalne oddziaływania: silne, słabe, elektromagnetyczne oraz grawitacyjne, każde z nich posiada kwanty pośredniczące. Dla oddziaływań silnych są to gluony, elektromagnetyczne są przenoszone przez fotony, słabe siły powodowane są wymianą naładowanych bozonów $W^{\pm}$ lub neutralnych $Z^{0}$. Jak przewiduje teoria, oddziaływania grawitacyjne są dozwolone dzięki grawitonom. Należy jednak pamiętać, że istnienie grawitonu jest, jak do tej pory, jedynie hipotezą niepotwierdzoną doświadczalnie. Poza wymienionymi cząstkami istnieje jeszcze pozbawiony spinu, koloru oraz ładunku elektrycznego bozon Higgsa odpowiedzialny za nadawanie masy innym cząstką.

Suma posiadanej wiedzy dotyczącej cząstek oraz ich wzajemnych oddziaływań jest zebrana w jedną teorią zwaną Modelem Standardowym (MS). Jednakże MS nie jest fundamentalną teorią. Zawiera parametry, które mogą być wyznaczone jedynie doświadczalnie, nie potrafi też odpowiedzieć na pytania dotyczące ciemnej materii czy oddziaływań grawitacyjnych.

Jednym z warunków koniecznych do wyjaśnienia obecnej asymetrii pomiędzy materią oraz antymaterią jest proces łamania symetrii kombinowanej CP. Zjawisko jest opisane w MS przez macierz Cabbibo-Kobayashiego-Masakawy(CKM). Jednakże wartości elementów tej macierzy nie są przewidziane przez MS, należy je wyznaczyć doświadczalnie. 
Jest to jeden z celów przyświecających powstaniu eksperymentu LHCb na akceleratorze LHC w ośrodku CERN, będącym największym laboratorium fizycznym na świecie. Jednym z kluczowych zadań, w celu którego został skonstruowany detektor jest pomiar pędu cząstek oddziałujących z jego powierzchnią czynna. Wykonanie tego zadania nie jest trywialne. W wyniku wieloletnich prac zaimplementowano szereg skomplikowanych algorytmów służących do znajdowania śladów, procesie niezbędnym do wyznaczania trzech składowych pędu cząstki naładowanej. 

Bardzo istotnym problemem, nadal występującym, jest iż otrzymywane wyniki nie są w pełni zadowalające. Studia nad jakością dopasowania śladów były przedmiotem badań wielu grup pracujących w ramach kolaboracji LHCb. Spowodowane jest to faktem, iż ewentualne błędy w wyznaczaniu pędu rzutują na wszystkie analizy wykonywane przez naukowców zrzeszonych w ramach kolaboracji LHCb.  
Niniejsza praca opisuje badania wykonane w ramach współpracy z Uniwersytetem Cincinnati, których głównym celem jest kontynuowanie analizy jakości dopasowania śladów. Wszystkie wyniki zgromadzone w tej pracy zostały zaprezentowane podczas spotkań roboczych grupy "Tracking and Alignment".



