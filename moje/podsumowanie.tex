\chapter*{Podsumowanie}
\addcontentsline{toc}{chapter}{Podsumowanie}

Głównym celem pracy było zbadanie jakości dopasowania śladów cząstek naładowanych w eksperymencie LHCb. W tym celu posłużono się testem $\chi^2$. Problem, który opisuje niniejsza praca jest był już poddawany badaniom przez kilka grupy badawczych między innymi z instytutu Nikhef w Amsterdamie, jednakże dalsze studia są niezbędne.  

Analiza została przeprowadzana przy użyciu dedykowanego programu napisanego przez autora, opartego na platformie programistycznej ROOT, w języku Python. Jak również rozszerzono standardowe oprogramowanie LHCb o dodatkowy moduł. 


W wyniku wykonanej analizy wyprodukowano szereg wykresów zarówno dla symulowanych danych Monte Carlo jak również dla rzeczywistych danych zebranych przez detektor LHCb. W wyniku badań  wydajność rekonstrukcji, ustalono jej zależność od pędu. Również stworzono wykresy rozkładów $\chi^2$ oraz korelacji pomiędzy nimi a parametrami takimi jak pęd, masa niezmienniczna czy pseudorapidity. Dla danych rzeczywistych wykonano ekstrakcję sygnału od szumu bazując na technice zwanej sPlot. Na podstawie tej ekstrakcji dokonano dalszej analizy jakości dopasowania tylko dla sygnału. Ustalono, że najbardziej negatywnie na jakość dopasowania śladów wpływa dodanie klastrów zmierzonych przez detektor TT. 

Całość prac wykonana została w ramach współpracy z uniwersytetem w Cincinnati. 
